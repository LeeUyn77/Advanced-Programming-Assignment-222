\section{Lời kết}
\begin{itemize}
    \item Khi chúng ta khám phá và áp dụng Design Pattern, chúng ta mở ra cánh cửa của một thế giới bất tận của kiến thức và công nghệ. Design Pattern không chỉ là một bộ công cụ giúp chúng ta giải quyết các vấn đề trong thiết kế phần mềm, mà nó còn là một nguồn cảm hứng, một hướng dẫn và một ngôn ngữ chung giữa các nhà phát triển.
    \item Qua việc áp dụng Design Pattern, chúng ta học cách tách rời và tái sử dụng các thành phần, tăng tính linh hoạt và dễ bảo trì của hệ thống. Chúng ta học cách xây dựng các kiến trúc mạnh mẽ, đảm bảo tính mở rộng và khả năng mở rộng trong tương lai.
    \item Tuy nhiên, cần lưu ý rằng Design Pattern không phải là một giải pháp tuyệt đối cho mọi vấn đề trong phát triển phần mềm. Chúng ta cần hiểu rõ bản chất của vấn đề và chọn lựa Design Pattern phù hợp để áp dụng. Đôi khi, việc quá sử dụng Design Pattern có thể làm cho hệ thống phức tạp hơn và khó hiểu hơn.
    \item Điều quan trọng nhất khi sử dụng Design Pattern là hiểu rõ mục đích và nguyên tắc đằng sau nó. Design Pattern không chỉ là một mô hình mã lệnh cụ thể, mà là một khái niệm thiết kế, một tư duy để giúp chúng ta đưa ra các quyết định thông minh và linh hoạt.
    \item Cuối cùng, việc học và áp dụng Design Pattern đòi hỏi thời gian và kỷ luật. Hãy sử dụng Design Pattern như một công cụ hỗ trợ, không phải là một rào cản trong quá trình phát triển phần mềm. Với sự am hiểu và sử dụng đúng cách, Design Pattern sẽ là một người bạn đồng hành đáng tin cậy trên con đường xây dựng phần mềm chất lượng cao.
\end{itemize}